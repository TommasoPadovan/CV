%%%%%%%%%%%%%%%%%%%%%%%%%%%%%%%%%%%%%%%%%
% Twenty Seconds Resume/CV
% LaTeX Template
% Version 1.1 (8/1/17)
%
% This template has been downloaded from:
% http://www.LaTeXTemplates.com
%
% Original author:
% Carmine Spagnuolo (cspagnuolo@unisa.it) with major modifications by 
% Vel (vel@LaTeXTemplates.com)
%
% License:
% The MIT License (see included LICENSE file)
%
%%%%%%%%%%%%%%%%%%%%%%%%%%%%%%%%%%%%%%%%%

%----------------------------------------------------------------------------------------
%	PACKAGES AND OTHER DOCUMENT CONFIGURATIONS
%----------------------------------------------------------------------------------------

\documentclass[letterpaper]{twentysecondcv} % a4paper for A4
\usepackage[utf8]{inputenc}
\usepackage{multicol}
\usepackage{enumitem}

\usepackage{bigfoot}
\DeclareNewFootnote[para]{A}
\expandafter\def\csname @makefnbreak\endcsname{\unskip\linebreak[0]\quad}


%----------------------------------------------------------------------------------------
%	 PERSONAL INFORMATION
%----------------------------------------------------------------------------------------

% If you don't need one or more of the below, just remove the content leaving the command, e.g. \cvnumberphone{}

\profilepic{images/ImgPado.png} % Profile picture

\cvname{Tommaso Padovan} % Your name
\cvjobtitle{Master's Student\\in Computer Science} % Job title/career

\cvdate{5 July 1993} % Date of birth
\cvaddress{Via Redipuglia 16\newline
Padova (PD), 35131\newline
Italy} % Short address/location, use \newline if more than 1 line is required
\cvnumberphone{+39 3484655214} % Phone number
\cvsite{} % Personal website
\cvmail{tommaso.pado@gmail.com} % Email address

%----------------------------------------------------------------------------------------

\begin{document}

%----------------------------------------------------------------------------------------
%	 ABOUT ME
%----------------------------------------------------------------------------------------

\aboutme{
	I put my hands on my first coding language when I was 13, developing simple video-games with \textit{DarkBASIC}, and I suddenly realized it was my way.\\
	Some years later I decided to "settle down", hang my games up for a while and begin studying computer science at \textit{Università degli Studi di Padova} (Italy). There I acquired strong knowledge in programming (mostly C++ and Java) and Software Engineering. In the same University I achieved my bachelor degree in 2016.\\
	After that I began my master and then moved to Darmstadt (Germany) to go "back to basics" and sharpen my skills in game technology, computer vision and hardware acceleration. 
} % To have no About Me section, just remove all the text and leave \aboutme{}

%----------------------------------------------------------------------------------------
%	 SKILLS
%----------------------------------------------------------------------------------------

% Skill bar section, each skill must have a value between 0 an 6 (float)
\skills{{French (B1)/2},{Italian (mother tongue)/6},{English (B2-C1)/4.7}}

%------------------------------------------------

% Skill text section, each skill must have a value between 0 an 6
\skillstext{}

%----------------------------------------------------------------------------------------

\makeprofile % Print the sidebar

%----------------------------------------------------------------------------------------
%	 INTERESTS
%----------------------------------------------------------------------------------------

% \section{Interests}

% The heroine and the dreamer of Wonderland; Alice is the principal character.

%----------------------------------------------------------------------------------------
%	 EDUCATION
%----------------------------------------------------------------------------------------

\section{Education}

\begin{twenty} % Environment for a list with descriptions
	\twentyitem{2016-now}
		{Master in Computer Science}	{Padova (Italy)}
		{At "Università degli Studi di Padova"\footnotemarkA[1]\newline
		And Erasmus at "TU Darmstadt"\footnotemarkA[2] (specializing in Computer Vision, CUDA and Game Technology)}
	\twentyitem{2013-2016}
		{Bachelor's degree in Computer Science}	{Padova (Italy)}
		{Achieved with 100/110 at "Università degli Studi di Padova"\footnotemarkA[2]\\
		Thesis title: "Last Tango In Mountain View"\footnotemarkA[3] regarding 3D reconstruction with computer vision techniques.}
	\twentyitem{2007-2012}
		{High school}	{Treviso (Italy)}
		{Lyceum of Sciences and Informatics}
	%\twentyitem{<dates>}{<title>}{<location>}{<description>}
\end{twenty}



%----------------------------------------------------------------------------------------
%	 PUBLICATIONS
%----------------------------------------------------------------------------------------

% \section{Publications}

% \begin{twentyshort} % Environment for a short list with no descriptions
% 	\twentyitemshort{1865}{Chapter One, Down the Rabbit Hole.}
% 	\twentyitemshort{1865}{Chapter Two, The Pool of Tears.}
% 	\twentyitemshort{1865}{Chapter Three,  The Caucus Race and a Long Tale.}
% 	\twentyitemshort{1865}{Chapter Four,  The Rabbit Sends a Little Bill.}
% 	\twentyitemshort{1865}{Chapter Five,  Advice from a Caterpillar.}
% 	%\twentyitemshort{<dates>}{<title/description>}
% \end{twentyshort}

%----------------------------------------------------------------------------------------
%	 AWARDS
%----------------------------------------------------------------------------------------

% \section{Awards}

% \begin{twentyshort} % Environment for a short list with no descriptions
% 	\twentyitemshort{1987}{All-Time Best Fantasy Novel.}
% 	\twentyitemshort{1998}{All-Time Best Fantasy Novel before 1990.}
% 	%\twentyitemshort{<dates>}{<title/description>}
% \end{twentyshort}

%----------------------------------------------------------------------------------------
%	 EXPERIENCE
%----------------------------------------------------------------------------------------

\section{Work Experience}

\begin{twenty} % Environment for a list with descriptions
	\twentyitem{2018-now}
		{Freelance at "EasyApp srl"\footnotemarkA[7]} {Treviso (Italy)}
		{Back-end for web and maintenance}
	\twentyitem{2016-2017}
		{Developer at "Vision Lab Apps srl"\footnotemarkA[4]}	{Vicenza (Italy)}
		{Android streaming application for smartphones and glasses.}
	\twentyitem{2016}
		{Developer at "Vic World Wide"\footnotemarkA[5] (Stage)}	{Padova (Italy)}
		{Prototype app for "world 3D reconstruction" using Google-Tango framework\footnotemarkA[6] on mobile devices.}
	\twentyitem{2014-2016}
		{Developer at "EasyApp srl"\footnotemarkA[7]}	{Treviso (Italy)}
		{Front-end and back-end for ERPs.}
	%\twentyitem{<dates>}{<title>}{<location>}{<description>}
\end{twenty}



%----------------------------------------------------------------------------------------
%	 OTHER INFORMATION
%----------------------------------------------------------------------------------------

\section{IT Skills}

\subsection{Software Development}

Good knowledge in \textit{procedural}, \textit{object-oriented}, \textit{concurrent} and \textit{functional} programming paradigms.\newline
Experience with both \textit{agile} and \textit{incremental} software development models.\newline
Good knowledge of the main Software Engineering topics: such as Design Pattern, SOLID principles, reusability and maintainability.\newline
Personal interest in Arduino/Raspberry projects.


\begin{minipage}[t]{0.45\linewidth}
	Strong experience with:
	\small
	\begin{itemize}[noitemsep]
		\item C / C++
		\item Scala (and akka)
		\item Java (and Android framework)
		\item Python
		\item php
	\end{itemize}
\end{minipage}
\begin{minipage}[t]{0.45\linewidth}
	Discreet experience with:
	\begin{itemize}[noitemsep,nolistsep]
		\item CUDA C++
		\item Javascript
		\item Kotlin
		\item Bash
		\item Julia
		\item Haskell
	\end{itemize}
\end{minipage}


\subsection{Tools and frameworks}
\begin{itemize}[noitemsep,nolistsep]
	\item Good practice with IDEs and frameworks as \textbf{JetBrains Suite}, \textbf{QT}, \textbf{Android Studio}, \textbf{Laravel}, etc.
	\item Familiarity with \textbf{Git}/\textbf{GitFlow} both with CLI and graphical interfaces; basic understanding of \textit{Subversion} and \textit{Mercurial}.
	\item Strong skills with \textbf{Android framework} and \textbf{Firebase suite}.
	\item Practice with database engines as \textbf{MySQL} and \textbf{Firebase Database}.
	\item Teamwork experience acquired during academic and professional projects; good skills in using work-management tools such as \textbf{teamwork.com}, \textbf{Trello}, \textbf{GanttProject}, \textbf{Slack}, etc.
	\item Good knowledge of \textbf{UML} graphical language and UML-editors (e.g. \textbf{Astah}).
	\item Some experience with game-engines such as \textbf{Unreal Engine}, \textbf{Unity} and \textbf{Kore}.
\end{itemize}




%----------------------------------------------------------------------------------------
%	 SECOND PAGE EXAMPLE
%----------------------------------------------------------------------------------------

\newpage % Start a new page

\normalsize
\makeprofile % Print the sidebar

\subsection{Some projects}
\small
\begin{itemize}[noitemsep,nolistsep]
	\item \textbf{Actobase:} NoSQL database based on the actor model written in Scala with akka. Developed in a group of seven students during an academic project. Source code available at: \url{https://github.com/SweeneyThreads/Actorbase}.
	\item \textbf{Baobab:} Small set of Python scripts for simple medical image analysis. Developed for/with a PhD in psychiatry. \url{https://github.com/TommasoPadovan/Baobab}
	\item \textbf{Other academic projects:} \\\url{https://github.com/TommasoPadovan/ProgettiUniPDTriennale}
\end{itemize}


\tiny
\footnotetextA[1]{\url{http://www.unipd.it}}
\footnotetextA[2]{\url{https://www.informatik.tu-darmstadt.de}}
\footnotetextA[3]{\url{https://github.com/TommasoPadovan/LastTangoInMountainView_TesiTriennale}(Italian)}
\footnotetextA[4]{\url{http://www.visionlabapps.com}}
\footnotetextA[5]{\url{http://vicworldwide.com}}
\footnotetextA[7]{\url{http://www.easyappsrl.it}}
\footnotetextA[6]{\url{https://developers.google.com/tango}}
%----------------------------------------------------------------------------------------

\end{document} 
